\documentclass[border=5]{standalone} 
\usepackage{tikz,array,centernot,amsmath,mathrsfs}
\setlength{\extrarowheight}{2em}
\tikzset{%
  v 0/.style={fill=white}, v 1/.style={fill=blue!30},
  pics/venn/.style args={#1#2#3#4}{code={%
    \fill [v #4/.try] (-2,-1.5) rectangle (2,1.5);
    \fill [v #3/.try] (90:sin 60) arc (120:-120:1) arc (-60:60:1);
    \fill [v #2/.try] (90:sin 60) arc (60:300:1)   arc (240:120:1);
    \fill [v #1/.try] (90:sin 60) arc (120:240:1)  arc (-60:60:1);
    \draw (-2,-1.5) rectangle (2,1.5)
      (90:sin 60) arc (120:-120:1) arc (-60:60:1)
      arc (60:300:1) arc (240:120:1) -- cycle;
}}}
\newcommand\venn[2][]{{\tikz[every venn/.try, #1]\pic{venn/.expanded=#2};}}
\tikzset{every venn/.style={x=1em, y=1em, baseline=-.666ex, 
  v 1/.style={fill=gray}}}
\begin{document} 
$\displaystyle
\begin{array}{|c|c|c|c|}
\hline
\textrm{Truth Table} & \textrm{Venn Diagram} & \textrm{Connective} & \textrm{Connective Name} \\
\hline
FFFF & \venn{0000} & \mathscr{P} \perp \mathscr{Q} & \textrm{Contradiction} \\
FFFT & \venn{0001} & \mathscr{P} \overline{\lor} \mathscr{Q} & \textrm{Nondisjunction (Nor)} \\
FFTF & \venn{0010} & \mathscr{P} \centernot\impliedby \mathscr{Q} & \textrm{Converse Nonimplication} \\[2em]
\hline
\end{array}
$
\end{document}