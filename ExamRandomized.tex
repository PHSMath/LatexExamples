\documentclass{exam}

\usepackage[utf8]{inputenc}
\usepackage[spanish,shorthands=off]{babel}
\usepackage{cmbright}
\usepackage[T1]{fontenc}
\usepackage{amsmath}
\usepackage{amsfonts}
\usepackage{amssymb}
\usepackage{amsthm}
\usepackage{tikz}
\usepackage{expl3}
\usepackage{environ}
\usetikzlibrary{calc, cd}

\newcommand{\n}[1]{\boldsymbol{#1}}

\theoremstyle{definition}
\newtheorem{ej}{Ejercicio}

\newcommand{\tq}{\;|\;}

\ExplSyntaxOn
\seq_new:N \l_luisturcio_questions_seq
\seq_new:N \l_luisturcio_indices_seq
\int_new:N \l_luisturcio_index_int
\int_new:N \l_luisturcio_questions_int
\tl_new:N \l_luisturcio_questions_tl
\tl_new:N \l_luisturcio_index_tl
\bool_new:N \l_luisturcio_loop_bool
\NewEnviron { randomizedquestions } [1][]
  {
    \luisturcio_sequence_questions:V \BODY
    \luisturcio_shuffle_questions:n { #1 }
    \luisturcio_create_questions_tl:
    \exp_args:Nno \begin{questions} \l_luisturcio_questions_tl \end{questions}
  }
\cs_new:Nn \luisturcio_sequence_questions:n
  {
    \seq_set_split:Nnn \l_luisturcio_questions_seq { \question } { #1 }
    \seq_remove_all:Nn \l_luisturcio_questions_seq {}
  }
\cs_generate_variant:Nn \luisturcio_sequence_questions:n { V }
\cs_new:Nn \luisturcio_shuffle_questions:n
  {
    \int_set:Nn \l_luisturcio_questions_int
      { \seq_count:N \l_luisturcio_questions_seq }
    \tl_if_empty:nTF { #1 }
      {
        \int_step_inline:nnnn
          { \l_luisturcio_questions_int } { -\c_one } { \c_one }
      }
      {
        \int_compare:nNnTF { \l_luisturcio_questions_int } > { #1 }
          {
            \int_step_inline:nnnn
              { \l_luisturcio_questions_int } { -\c_one }
              { \l_luisturcio_questions_int - #1 + \c_one }
          }
          {
            \int_step_inline:nnnn
              { \l_luisturcio_questions_int } { -\c_one } { \c_one }
          }
      }
      {
        \int_set:Nn \l_luisturcio_index_int { \int_rand:nn { \c_one } { ##1 } }
        \luisturcio_add_to_indices:
      }
  }
\cs_new:Nn \luisturcio_add_to_indices:
  {
    \bool_set_true:N \l_luisturcio_loop_bool
    \bool_do_while:Nn \l_luisturcio_loop_bool
      {
        \seq_if_in:NxTF \l_luisturcio_indices_seq
          { \int_use:N \l_luisturcio_index_int }
          {
            \int_incr:N \l_luisturcio_index_int
          }
          {
            \seq_put_right:Nx \l_luisturcio_indices_seq
              { \int_use:N \l_luisturcio_index_int }
            \bool_set_false:N \l_luisturcio_loop_bool
          }
      }
  }
\cs_new:Nn \luisturcio_create_questions_tl:
  {
    \tl_clear:N \l_luisturcio_questions_tl
    \bool_while_do:nn { ! \seq_if_empty_p:N \l_luisturcio_indices_seq }
      {
        \tl_put_right:Nn \l_luisturcio_questions_tl { \question }
        \seq_pop_right:NN \l_luisturcio_indices_seq \l_luisturcio_index_tl
        \tl_put_right:Nx \l_luisturcio_questions_tl
          {
            \seq_item:Nn \l_luisturcio_questions_seq { \l_luisturcio_index_tl }
          }
      }
  }
\ExplSyntaxOff

\title{Conjuntos Abstractos}
\author{}
\date{}


\begin{document}

\maketitle

\begin{randomizedquestions}
\question Demuestra que iso implica biyectiva.

\question Demuestra que la categoría $1/\mathcal{S}$ tiene todos los límites finitos. Los objetos de $1/\mathcal{S}$ son parejas $(A,a:1\to A)$, donde $A$ es un conjunto abstracto; y las flechas $f:(A,a:1\to A)\to(B,b:1\to B)$ son flechas entre conjuntos abstractos $f:A\to B$ que hacen conmutar al siguiente diagrama
\begin{center}
    \begin{tikzcd}[ampersand replacement=\&]
        \& 1 \arrow[ld, "a"swap] \arrow[rd, "b"] \\
        A \arrow[rr, "f"swap] \&\& B
    \end{tikzcd}
\end{center}

\question Demuestra que si una categoría $\n{C}$ tiene coproductos y coigualadores entonces tiene todos los colímites.

\question Demuestra que todo igualador es mono.

\question Sean \tikzcd[cramped, sep=small,ampersand replacement=\&] U \arrow[r, rightarrowtail, "i"] \& A \endtikzcd{} e \tikzcd[cramped, sep=small,ampersand replacement=\&] V \arrow[r, rightarrowtail, "j"] \& A \endtikzcd{} subobjetos de $A$. Diremos que $i$ es equivalente ($i\sim j$) a $j$ si $i\subseteq j$ y $j\subseteq i$. Demuestra que la relación $\sim$ es de equivalencia.

\question Sean $A$ un conjuntos abstractos y considera los conjuntos
\begin{gather*}
Sub(A)=\{\tikzcd[cramped, sep=small, ampersand replacement=\&] U \arrow[r, rightarrowtail, "i"] \& A \endtikzcd{}
         \tq \text{$i$ es mono}\}/\sim \\
\mathcal{S}(A,B)=\{f:A\to B\tq\text{$f$ es una flecha en $\mathcal{S}$}\}
\end{gather*}
Demuestra que hay una biyección $Sub(A)\cong\mathcal{S}(A,2)$.
\end{randomizedquestions}

\end{document}