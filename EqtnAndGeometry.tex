\documentclass[a4paper]{article}
\usepackage[scale=.85]{geometry}
\usepackage{tikz}
\usetikzlibrary{shapes.geometric}
\newcounter{tikzqn}
\newcounter{tikzrow}
\newenvironment{tikzqn}%
{%
  \stepcounter{tikzqn}%
  \stepcounter{tikzrow}%
  \begin{minipage}[t]{.3\textwidth}
    \sffamily\thetikzqn)\par\centering
    \begin{tikzpicture}[baseline=(current bounding box.north), font=\sffamily, thick]
}{%
    \end{tikzpicture}
    \bigskip\vfill\par
  \end{minipage}\hfill
  \ifnum\value{tikzrow}=3\setcounter{tikzrow}{0}%
  \bigskip\par
  \foreach \i in {1,2,3}
  {\begin{minipage}{.3\textwidth}
    \sffamily Answer:\hrulefill\bigskip\par
  \end{minipage}\hfill}%
\fi}
\pagestyle{empty}
\usepackage{cabin}


\begin{document}
Find the following areas:\\
\tikzset{
  pics/my circle/.style 2 args={
    code={
      \draw circle (#1);
      \draw [fill] (0,0) circle (1pt) -- (#1,0) node [midway, above] {#2};
    }
  },
}
\noindent
\begin{tikzqn}
   \pic {my circle={20mm}{71cm}};
\end{tikzqn}
\begin{tikzqn}
  \draw (0,0) -- ++(40mm,0) node [midway, below] {37cm} -- ++(0,20mm) node [midway, right, anchor=north, sloped] {21cm} -| cycle;
\end{tikzqn}
\begin{tikzqn}
  \draw (0,0) coordinate (a) -- ++(20mm,0) node [midway, below] {10cm} -- ++(-10mm,25mm) coordinate (c) -- cycle;
  \draw [dashed] (a -| c) -- (c) node [pos=.35, right, anchor=south, sloped] {8cm};
  \draw (a -| c) ++ (0,5pt) -| ++(5pt,-5pt);
\end{tikzqn}

\begin{tikzqn}
  \draw (0,0) -- ++(0,40mm) node [midway, right, anchor=south, sloped] {3m} -| ++(10mm,-40mm) -- cycle node [midway, below, anchor=north] {75cm};
\end{tikzqn}
\begin{tikzqn}
  \pic {my circle={15mm}{4cm}};
\end{tikzqn}
\begin{tikzqn}
  \draw (0,0) -| ++(20mm,25mm) coordinate [midway] (b) coordinate (c) -- cycle;
  \path (0,0) -- (b) node [midway, below, anchor=north] {11cm} -- (c) node [midway, right, anchor=north, sloped] {17cm};
  \draw (b) rectangle ++(-5pt,5pt);
\end{tikzqn}

\begin{tikzqn}
  \node (dia) [draw, shape=diamond, minimum size=30mm] {};
  \node [rotate=45, anchor=north] at (dia.south east)  {6cm};
  \draw (dia.north) ++(-2.5pt,-2.5pt) -- ++(2.5pt,-2.5pt) -- ++(2.5pt,2.5pt);
  \draw (dia.south) ++(-2.5pt,2.5pt) -- ++(2.5pt,2.5pt) -- ++(2.5pt,-2.5pt);
\end{tikzqn}
\begin{tikzqn}
  \node (sc) [draw, shape=semicircle, minimum size=25mm, label=below:46mm] {};
  \draw [dashed] (sc.south) -- (sc.north) node [midway, left, anchor=south, sloped] {23mm};
  \draw (sc.south) ++(5pt,0) |- ++(-5pt,5pt);
\end{tikzqn}
\begin{tikzqn}
  \draw (0,0) -- ++(35mm,0) coordinate (a) node [midway, below, anchor=north] {14cm} arc (-90:90:15mm) coordinate (b) -| cycle;
  \draw [dashed] (a -| b) -- (b) node [midway, right, anchor=south, sloped] {5cm};
  \draw [dashed] ([yshift=15mm]a) coordinate (c) -- ++(15mm,0) node [midway, below, anchor=north] {2.5cm};
  \draw (a) ++(-5pt,0) |- ++(5pt,5pt) (c) ++(5pt,0) |- ++(-5pt,5pt);
\end{tikzqn}

\begin{tikzqn}
  \node (tri) [draw, regular polygon, regular polygon sides=3, label=-90:10cm, minimum size=45mm] {};
  \path (tri.corner 2) -- (tri.corner 1) node [midway, left, anchor=south, sloped] {10cm};
  \path (tri.corner 3) -- (tri.corner 1) node [midway, left, anchor=south, sloped] {10cm};
\end{tikzqn}
\begin{tikzqn}
  \node (sq) [draw, regular polygon, regular polygon sides=4, label=-90:8cm, minimum size=40mm] {};
  \path (sq.south east) -- (sq.north east) node [midway, right, sloped, anchor=north] {8cm};
\end{tikzqn}
\begin{tikzqn}
  \node [draw, regular polygon, regular polygon sides=5, label=-90:7cm, minimum size=35mm] {};
\end{tikzqn}

\end{document}